\documentclass[11pt,a4paper]{article}
\usepackage[utf8]{inputenc}
\usepackage[T1]{fontenc}
\usepackage[francais]{babel}
%\usepackage{fullpage}
\usepackage{gensymb}

\begin{document}

Page de garde

\newpage

\section{Présentation du projet}

\subsection{Objectif général}
L'objectif général de notre projet de fin d'études, dit \textit{PFE}, est d'étudier et de mettre en place une chaîne de transmission vidéo.
Nous ferons une étude technique et la réalisation de l'ensemble des étapes : de la station émettrice au récepteur, en passant par les techniques de mise en réseau.

\bigbreak
Nous avons décidé de nous intéresser plus particulièrement à l'amélioration de la qualité d'une zone géographique d'un flux vidéo, au détriment du reste de la vidéo.

Ce principe peut se révéler intéressant dans des cas où le consommateur d'un flux vidéo ne serait intéressé que par une zone précise d'une image capturée par une caméra, tout en étant susceptible de changer de zone d'intérêt à tout instant.
La zone d'intérêt est choisie et commandée par le client récepteur du flux vidéo, afin que l'émetteur puisse adapter la zone d’intérêt pour la suite de la transmission.

\bigbreak
Nous essaierons de faire des choix technologiques les plus simples et compréhensibles possibles et d'assurer une interopérabilité entre différentes plateformes matérielles et logicielles.

\subsection{Cas d'application}

Afin d'illustrer ce système en pratique, nous l'implémenterons au sein d'une solution de transmission vidéo distante en réalité virtuelle.
Une station émettrice mobile capture un flux vidéo provenant d'une caméra à grand angle et transmet un flux qui sera alors reçu sur un autre terminal mobile n'affichant qu'une zone précise de l'image à travers une solution de réalité virtuelle de type Google Cardboard.
Le terminal est donc déplacé avec les mouvements de tête du spectateur et la zone de visionnage est donc déplacée en conséquence pour obtenir un effet immersif.
La nouvelle région d’intérêt est renvoyée du terminal mobile à l'émetteur pour faire à nouveau coïncider la zone visionnée avec la région d'intérêt à qualité améliorée.

\subsection{Contraintes qualitatives}

Nous avons la contrainte de transmettre l’ensemble de la vidéo pour pouvoir répondre instantanément aux déplacements de la tête de spectateur.
En effet, des solutions existantes comme celle choisie par certains drones tels que celui de Parrot font le choix de ne transmettre que la région d’intérêt. Cela présente l’inconvénient de devoir attendre un aller/retour complet de transmission avant d’avoir l’image demandée. Pour un effet immersif complet, cela ne peut pas être envisagé.
D’autres solutions font le choix de simplement transmettre l’image complète non-modifiée, assurant ainsi une image de qualité moyenne sur l’ensemble de la scène.

\bigbreak
Notre solution devra donc permettre d’avoir une qualité supérieure de la zone observée au prix d’une dégradation du reste de l’image.
Ainsi, lorsque le spectateur bouge la tête, il voit immédiatement la partie de l’image correspondante, mais en qualité dégradée. Il faudra attendre l’aller/retour complet pour obtenir une image de bonne qualité.

\bigbreak
Pour pouvoir mesurer une amélioration de la qualité et s’adapter à diverses conditions de transmissions réseau, nous utiliserons un encodeur à débit contraint, de telle sorte que ce soit la qualité qui soit variable.
Ce débit pourra d’ailleurs être adapté en fonction des pertes réseaux observées par le récepteur.


\section{Recherche}

Nous avons sélectionné quelques papiers qui s'intéressent à des techniques s'approchant de notre projet : la compression vidéo avec une qualité variable en fonction de la zone d'intérêt de l'utilisateur.

\bigbreak
La solution la plus évidente est d'intervenir directement au niveau de l'algorithme de compression.
L'article de Amir Said et William Pearlman\ref{REF1} décrit comment compresser une image avec un débit variable pour chaque partie de l'image.

Cette approche est intéressante mais nous faisons le choix de réutiliser un algorithme de compression, dit \textit{codec}, existant.
L'algorithme étant notamment déjà intégré dans les terminaux de réception pour effectuer du décodage matériel, il nous est presque impossible d'intervenir à ce niveau.

\bigbreak
L'article de Engin Kurutepe\ref{REF2} montre comment optimiser l'envoi d'un flux provenant de plusieurs caméras.
Pour cela, l'équipement récepteur suit les mouvements de la tête de l'utilisateur afin de savoir quelle zone de l'image l'intéresse le plus.
Après cela, l'équipement envoie cette information à l'encodageur qui va demander un plus gros débit aux caméras qui filment la zone d'intérêt et un débit plus faible aux autres caméras.

Nous ne pouvons pas réutiliser ce concept car nous n'avons qu'une seule caméra qui filme la scène et le principe décrit par l'article est applicable seulement pour une même scène filmée avec plusieurs caméras.
Cela nous a tout de même confortés dans l'idée d'encoder une vidéo sphérique avec une qualité dynamique, mais à l'intérieur d'une même image.

\bigbreak
Enfin, l'article de Aditya Mavlankar et Bernd Girod\ref{REF3} explique comment optimiser l’envoi d’une image basse qualité d’un match de football ainsi qu’une zone précise de haute qualité.
L’idée ressemble à ce que nous voulons faire, mais les recherches se portent sur l’optimisation de l’encodage pour du broadcast ; comment éviter d’encoder le flux de haute qualité pour chaque téléspectateur.
Or, notre projet fonctionne dans un modèle mono-utilisateur, il sera donc plus simple et plus efficace pour notre modèle d’encoder le flux entier en prenant directement en compte la zone d'intérêt de l'utilisateur.
Aussi, cet article cite les mécanismes FMO et ASO du standard de compression vidéo H.264 qui permettent respectivement d’avoir un débit différent pour chaque bloc de l'image et de prioriser la qualité des zones de l’image.
Malheureusement, ces plugins ne sont pas implémentés par défaut dans des outils libres comme FFmpeg ou la libx264 et peu de récepteurs mobiles ne les comprennent.

\bigbreak
Globalement, nous n'avons pas pu trouver d'articles décrivant un procédé permettant d'implémenter directement notre projet, nous partirons donc sur l'étude et la conception d'une solution propre.
Cette recherche nous aura tout de même permis d'avoir un bon aperçu des techniques de compression vidéo couramment utilisées.


\section{Architecture générale}

\section{Détails d'implémentation}

\subsection{Acquisition d'image RAW}
Pour notre projet, nous nous sommes procurés une caméra à grand angle (180\degree) pour être en mesure de filmer l'intégralité d'une scène.
\bigbreak
-> Photo Caméra
\bigbreak
La caméra est compatible avec le driver générique \textit{uvcvideo} inclus par défaut dans le noyau Linux, et peut donc être utilisée à travers \textit{Video4Linux 2}, V4L2.

\bigbreak
Afin d'obtenir l'image la moins altérée possible et de faciliter les futurs traitements sur l'image, nous récupérons les images RAW non-compressées.
La caméra est capable, à travers l'USB2, de tenir un flux de données correspondant à des images d'une résolution de 1280x720 à une fréquence de 10 images par seconde.
% Détail sur le débit en sortie du port USB ?

\bigbreak
L'acquisition est réalisée grâce à notre première instance de FFmpeg et à son module d'acquisition V4L2 qui permettent d'obtenir les octets correspondants sur la sortie standard pour la suite de notre traitement.

\bigbreak
La caméra utilisée présente une particularité : le capteur CMOS utilisé est un capteur standard et l'effet grand angle est obtenu par l'apposition d'une lentille convexe au-dessus.
Si cela permet alors d'obtenir un champ de vision très proche de 180\degree, l'image en retour présente un effet \textit{fisheye} non-négligeable.

\bigbreak
-> Capture effet fisheye brut
\bigbreak

Pour rendre des proportions réalistes aux abords de l'image, il est nécessaire de procéder à une correction de la déformation induite par la lentille.
Heureusement, FFmpeg dispose d'un certain nombre de plugins de traitement d'image que nous pouvons utiliser pendant la phase d'acquisition d'image RAW.

\bigbreak
L'un d'entre eux, issu de la collection de plugins \textit{Frei0r}, est le plugin \textit{Defish0r}, relativement simple d'utilisation et qui permet d'annuler cet effet de distorsion.
Malheureusement, les premiers tests ont montré que le plugin était trop gourmand en calcul CPU, et qu'il était impossible de traiter 10 images par seconde : le retard s'accumule rapidement.

\bigbreak
Un autre plugin, plus générique, est présent dans FFmpeg : \textit{lenscorrection}.
Plus léger en calcul, il est capable de traiter nos 10 images par seconde avec notre carte Raspberry Pi 3.
Son inconvénient est son paramétrage : le filtre demande deux coefficient correspondant aux termes de second et quatrième degrés dans la formule de correction :

$$r_{src} = r_{targ} * (1 + k1*(\frac{r_{targ}}{r_0})^2 + k2*(\frac{r_{targ}}{r_0})^4)$$
avec :
\begin{itemize}
\item{$k1$, $k2$ : coefficients considérés}
\item{$r_0$ : moitié de la diagonale de l'image}
\item{$r_{src}$ : distance entre le point considéré dans l'image source et le centre de l'image}
\item{$r_{targ}$ : distance du point à placer dans l'image de destination avec le centre de l'image}
\end{itemize}

\bigbreak

La difficulté réside donc dans la détermination des coefficients $k1$ et $k2$ adaptés à notre configuration capteur+lentille.
Les valeurs se situant dans l'intervalle $[-1;1]$, nous réalisons un test visuel en générant des images avec tous les couples de $(k1, k2)$ possibles en prenant un pas de $p=0.1$, soit une série de 441 images.

\bigbreak
-> Mosaique images ?

\bigbreak
Le détermination des coefficients s'est donc faite de façon qualitative en visionnant la série d'images et en choisissant l'image présentant le moins de déformation tout en évitant les aberrations.
Les valeurs suivantes semblent correspondre à notre lentille :
$$k1 = -0.4, k2 = 0.1$$

\subsection{Traitement de l'image YUV}
Si nous avons maintenant une image brute corrigée, nous souhaitons lui appliquer notre traitement spécifique.

\bigbreak
L'image RAW d'entrée est donc d'une résolution de 1280x720 avec un codage des pixels selon le format \texttt{YUV 4:2:0}.
Ce format est très utilisé dans le codage de la vidéo car il 

\subsection{Android}
Pour la partie réception mobile, nous avons choisi d'utiliser un smartphone Android comme support de développement.
En effet, le kit de développement offert par Google est gratuit, disponible sur des appareils mobiles et avec une puissance assez élevée pour effectuer du traitement vidéo en direct.
De plus, depuis quelques années, Google développe un écosystème de réalité de virtuelle pour Android.

\bigbreak
Nous avons donc dévellopé une application Android en Java, qui récupère le flux video en Wi-Fi, décompresse la vidéo et l'affiche sur l'écran pour un casque de type Cardboard.
Cette application comporte une activité qui démarre la partie réception video et affiche le flux vidéo. De plus, cette activité possède un menu qui permet à l'utilisateur d'activer ou non le traitement vidéo et de définir sa source vidéo (une une vidéo de test embarquée dans l'application ou un flux vidéo distant).
Nous avons implémenté deux récepteurs vidéo différents mais qui implémente la même interface Java afin d'être facilement interchangeable par l'activité qui les lance.

\bigbreak
Le premier récepteur est une version simplifée qui utilise une vidéo stockée en local dans l'application afin de tester la partie traitement vidéo sans avoir à utiliser un émetteur.
Techniquement, c'est un thread qui va regulierement lire le flux vidéo et l'envoyer au décodeur vidéo. 
Pour la partie lecture du flux vidéo, nous utilisons un exctractor (class fournit par le SDK Android) qui permet de parser un fichier vidéo. Nous pouvons ensuite demander à l'extrator le type de video contenu dans le fichier, puis d'écrire une partie partie du flux vidéo dans un buffer.
Pour décoder ce flux vidéo, nous utilisons l'API MediaCodec qui permet d'accéder au support multimédia bas-niveau fournit par le SDK. Ce MediaCodec permet, entre autres, de décoder un flux vidéo.
Pour fonctionner, nous configurons un mediaCodec avec les paramètres de la vidéo trouvés via l'extractor ainsi que la surface dans laquelle la vidéo décodés doit être envoyés (cette partie sera expliqué dans la sous-section suivante).
Puis, grâce à notre thread, nous allons régulièrement demander au mediaCodec de nous fournir un buffer, que nous transferons à l'extractor afin qu'il ajoute les données suivantes du flux vidéo.
Ce mecanisme de réutilsiation de buffers, gérés par MediaCodec, permet au développeur de ne pas avoir a s'occuper de l'allocation de gros volumes de données (qui pourrait être une source de perte de performance). 

\bigbreak
La deuxième implémentatino

\section{Résultats}

\section{Conclusion}

\end{document}
